\documentclass[11pt,]{article}
\usepackage[left=1in,top=1in,right=1in,bottom=1in]{geometry}
\newcommand*{\authorfont}{\fontfamily{phv}\selectfont}
\usepackage[]{mathpazo}


  \usepackage[T1]{fontenc}
  \usepackage[utf8]{inputenc}




\usepackage{abstract}
\renewcommand{\abstractname}{}    % clear the title
\renewcommand{\absnamepos}{empty} % originally center

\renewenvironment{abstract}
 {{%
    \setlength{\leftmargin}{0mm}
    \setlength{\rightmargin}{\leftmargin}%
  }%
  \relax}
 {\endlist}

\makeatletter
\def\@maketitle{%
  \newpage
%  \null
%  \vskip 2em%
%  \begin{center}%
  \let \footnote \thanks
    {\fontsize{18}{20}\selectfont\raggedright  \setlength{\parindent}{0pt} \@title \par}%
}
%\fi
\makeatother




\setcounter{secnumdepth}{0}

\usepackage{color}
\usepackage{fancyvrb}
\newcommand{\VerbBar}{|}
\newcommand{\VERB}{\Verb[commandchars=\\\{\}]}
\DefineVerbatimEnvironment{Highlighting}{Verbatim}{commandchars=\\\{\}}
% Add ',fontsize=\small' for more characters per line
\usepackage{framed}
\definecolor{shadecolor}{RGB}{248,248,248}
\newenvironment{Shaded}{\begin{snugshade}}{\end{snugshade}}
\newcommand{\AlertTok}[1]{\textcolor[rgb]{0.94,0.16,0.16}{#1}}
\newcommand{\AnnotationTok}[1]{\textcolor[rgb]{0.56,0.35,0.01}{\textbf{\textit{#1}}}}
\newcommand{\AttributeTok}[1]{\textcolor[rgb]{0.77,0.63,0.00}{#1}}
\newcommand{\BaseNTok}[1]{\textcolor[rgb]{0.00,0.00,0.81}{#1}}
\newcommand{\BuiltInTok}[1]{#1}
\newcommand{\CharTok}[1]{\textcolor[rgb]{0.31,0.60,0.02}{#1}}
\newcommand{\CommentTok}[1]{\textcolor[rgb]{0.56,0.35,0.01}{\textit{#1}}}
\newcommand{\CommentVarTok}[1]{\textcolor[rgb]{0.56,0.35,0.01}{\textbf{\textit{#1}}}}
\newcommand{\ConstantTok}[1]{\textcolor[rgb]{0.00,0.00,0.00}{#1}}
\newcommand{\ControlFlowTok}[1]{\textcolor[rgb]{0.13,0.29,0.53}{\textbf{#1}}}
\newcommand{\DataTypeTok}[1]{\textcolor[rgb]{0.13,0.29,0.53}{#1}}
\newcommand{\DecValTok}[1]{\textcolor[rgb]{0.00,0.00,0.81}{#1}}
\newcommand{\DocumentationTok}[1]{\textcolor[rgb]{0.56,0.35,0.01}{\textbf{\textit{#1}}}}
\newcommand{\ErrorTok}[1]{\textcolor[rgb]{0.64,0.00,0.00}{\textbf{#1}}}
\newcommand{\ExtensionTok}[1]{#1}
\newcommand{\FloatTok}[1]{\textcolor[rgb]{0.00,0.00,0.81}{#1}}
\newcommand{\FunctionTok}[1]{\textcolor[rgb]{0.00,0.00,0.00}{#1}}
\newcommand{\ImportTok}[1]{#1}
\newcommand{\InformationTok}[1]{\textcolor[rgb]{0.56,0.35,0.01}{\textbf{\textit{#1}}}}
\newcommand{\KeywordTok}[1]{\textcolor[rgb]{0.13,0.29,0.53}{\textbf{#1}}}
\newcommand{\NormalTok}[1]{#1}
\newcommand{\OperatorTok}[1]{\textcolor[rgb]{0.81,0.36,0.00}{\textbf{#1}}}
\newcommand{\OtherTok}[1]{\textcolor[rgb]{0.56,0.35,0.01}{#1}}
\newcommand{\PreprocessorTok}[1]{\textcolor[rgb]{0.56,0.35,0.01}{\textit{#1}}}
\newcommand{\RegionMarkerTok}[1]{#1}
\newcommand{\SpecialCharTok}[1]{\textcolor[rgb]{0.00,0.00,0.00}{#1}}
\newcommand{\SpecialStringTok}[1]{\textcolor[rgb]{0.31,0.60,0.02}{#1}}
\newcommand{\StringTok}[1]{\textcolor[rgb]{0.31,0.60,0.02}{#1}}
\newcommand{\VariableTok}[1]{\textcolor[rgb]{0.00,0.00,0.00}{#1}}
\newcommand{\VerbatimStringTok}[1]{\textcolor[rgb]{0.31,0.60,0.02}{#1}}
\newcommand{\WarningTok}[1]{\textcolor[rgb]{0.56,0.35,0.01}{\textbf{\textit{#1}}}}



\title{Programación Estadística: Funciones  }



\author{\Large Adrián Sosa\vspace{0.05in} \newline\normalsize\emph{}   \and \Large \vspace{0.05in} \newline\normalsize\emph{Universidad Veracruzana}  }


\date{}

\usepackage{titlesec}

\titleformat*{\section}{\normalsize\bfseries}
\titleformat*{\subsection}{\normalsize\itshape}
\titleformat*{\subsubsection}{\normalsize\itshape}
\titleformat*{\paragraph}{\normalsize\itshape}
\titleformat*{\subparagraph}{\normalsize\itshape}


\usepackage{natbib}
\bibliographystyle{plainnat}
\usepackage[strings]{underscore} % protect underscores in most circumstances



\newtheorem{hypothesis}{Hypothesis}
\usepackage{setspace}


% set default figure placement to htbp
\makeatletter
\def\fps@figure{htbp}
\makeatother

\usepackage{hyperref}

% move the hyperref stuff down here, after header-includes, to allow for - \usepackage{hyperref}

\makeatletter
\@ifpackageloaded{hyperref}{}{%
\ifxetex
  \PassOptionsToPackage{hyphens}{url}\usepackage[setpagesize=false, % page size defined by xetex
              unicode=false, % unicode breaks when used with xetex
              xetex]{hyperref}
\else
  \PassOptionsToPackage{hyphens}{url}\usepackage[draft,unicode=true]{hyperref}
\fi
}

\@ifpackageloaded{color}{
    \PassOptionsToPackage{usenames,dvipsnames}{color}
}{%
    \usepackage[usenames,dvipsnames]{color}
}
\makeatother
\hypersetup{breaklinks=true,
            bookmarks=true,
            pdfauthor={Adrián Sosa () and  (Universidad Veracruzana)},
             pdfkeywords = {},  
            pdftitle={Programación Estadística: Funciones},
            colorlinks=true,
            citecolor=blue,
            urlcolor=blue,
            linkcolor=magenta,
            pdfborder={0 0 0}}
\urlstyle{same}  % don't use monospace font for urls

% Add an option for endnotes. -----


% add tightlist ----------
\providecommand{\tightlist}{%
\setlength{\itemsep}{0pt}\setlength{\parskip}{0pt}}

% add some other packages ----------

% \usepackage{multicol}
% This should regulate where figures float
% See: https://tex.stackexchange.com/questions/2275/keeping-tables-figures-close-to-where-they-are-mentioned
\usepackage[section]{placeins}


\begin{document}
	
% \pagenumbering{arabic}% resets `page` counter to 1 
%
% \maketitle

{% \usefont{T1}{pnc}{m}{n}
\setlength{\parindent}{0pt}
\thispagestyle{plain}
{\fontsize{18}{20}\selectfont\raggedright 
\maketitle  % title \par  

}

{
   \vskip 13.5pt\relax \normalsize\fontsize{11}{12} 
\textbf{\authorfont Adrián Sosa} \hskip 15pt \emph{\small }   \par \textbf{\authorfont } \hskip 15pt \emph{\small Universidad Veracruzana}   

}

}






\vskip -8.5pt


 % removetitleabstract

\noindent  

\hypertarget{funciones}{%
\section{Funciones}\label{funciones}}

\hypertarget{documentaciuxf3n-de-funciones}{%
\subsection{Documentación de
Funciones}\label{documentaciuxf3n-de-funciones}}

La ayuda en línea de R proporciona información muy útil sobre cómo
utilizar las funciones. La ayuda se encuentra disponible directamente
para una función dada por medio de comandos o en la sección help.

Hay diferentes maneras de obtener la ayuda de una función, por comandos
o en la sección help con el nombre de la función. Los comandos
utilizados son dos \emph{?} y \emph{help()}, al ejecutar las funciones
de ayuda se abrira el menú de ayuda con el titulo de la función seguido
de mas información como una breve descripción, como se usa y los
argumentos que acepta, ejemplos entre otros.

Un ejemplo de uso seria el siguiente, para la función \emph{lm()}
(modelo lineal):

\begin{Shaded}
\begin{Highlighting}[]
\CommentTok{# ?lm}

\CommentTok{# help(lm)}
\end{Highlighting}
\end{Shaded}

\hypertarget{importaciuxf3n-de-datos}{%
\subsection{Importación de datos}\label{importaciuxf3n-de-datos}}

\hypertarget{read.csv}{%
\subsubsection{read.csv()}\label{read.csv}}

Esta función lee un archivo en formato CSV y crea un data frame.

El modo de usarlo es el siguiente:

\begin{verbatim}
read.csv(file, header = TRUE, sep = ",", quote = "\"", dec = ".", fill = TRUE, ...)
\end{verbatim}

\begin{Shaded}
\begin{Highlighting}[]
\CommentTok{# se crea un archivo temporal}
\NormalTok{test1 <-}\StringTok{ }\KeywordTok{c}\NormalTok{(}\DecValTok{1}\OperatorTok{:}\DecValTok{5}\NormalTok{, }\StringTok{"6,7"}\NormalTok{, }\StringTok{"8,9,10"}\NormalTok{)}
\NormalTok{tf <-}\StringTok{ }\KeywordTok{tempfile}\NormalTok{()}
\KeywordTok{writeLines}\NormalTok{(test1, tf)}

\CommentTok{# lectura de csv}
\KeywordTok{read.csv}\NormalTok{(tf, }\DataTypeTok{fill =} \OtherTok{TRUE}\NormalTok{)}
\end{Highlighting}
\end{Shaded}

\begin{verbatim}
##   X1
## 1  2
## 2  3
## 3  4
## 4  5
## 5  6
## 6  7
## 7  8
## 8  9
## 9 10
\end{verbatim}

\hypertarget{data.frame}{%
\subsubsection{data.frame()}\label{data.frame}}

La función \emph{data.frame()} crea un marco de datos, colecciones de
variables acopladas que comparten atributos de las matrices y listas.

El modo de usarlo es el siguiente:

\begin{verbatim}
data.frame(..., row.names = NULL, check.rows = FALSE,check.names = TRUE)
\end{verbatim}

\begin{Shaded}
\begin{Highlighting}[]
\CommentTok{# se crea un vector de letras}
\NormalTok{L3 <-}\StringTok{ }\NormalTok{LETTERS[}\DecValTok{1}\OperatorTok{:}\DecValTok{3}\NormalTok{]}
\NormalTok{fac <-}\StringTok{ }\KeywordTok{sample}\NormalTok{(L3, }\DecValTok{10}\NormalTok{, }\DataTypeTok{replace =} \OtherTok{TRUE}\NormalTok{)}
\CommentTok{# se crea el data frame con tres columnas}
\NormalTok{d <-}\StringTok{ }\KeywordTok{data.frame}\NormalTok{(}\DataTypeTok{x =} \DecValTok{1}\NormalTok{, }\DataTypeTok{y =} \DecValTok{1}\OperatorTok{:}\DecValTok{10}\NormalTok{, }\DataTypeTok{fac =}\NormalTok{ fac)}
\NormalTok{d}
\end{Highlighting}
\end{Shaded}

\begin{verbatim}
##    x  y fac
## 1  1  1   A
## 2  1  2   B
## 3  1  3   B
## 4  1  4   C
## 5  1  5   A
## 6  1  6   A
## 7  1  7   B
## 8  1  8   A
## 9  1  9   C
## 10 1 10   A
\end{verbatim}

\hypertarget{write.csv}{%
\subsubsection{write.csv()}\label{write.csv}}

La función \emph{data.frame()} crea un marco de datos, colecciones de
variables acopladas que comparten atributos de las matrices y listas.

El modo de usarlo es el siguiente:

\begin{verbatim}
write.csv(data, file="file_name.csv")
\end{verbatim}

\begin{Shaded}
\begin{Highlighting}[]
\CommentTok{# se crea un data.frame()}
\NormalTok{x <-}\StringTok{ }\KeywordTok{data.frame}\NormalTok{(}\DataTypeTok{a =} \KeywordTok{I}\NormalTok{(}\StringTok{"a }\CharTok{\textbackslash{}"}\StringTok{ quote"}\NormalTok{), }\DataTypeTok{b =}\NormalTok{ pi)}

\CommentTok{# se guarda el archivo en la ruta especificada}
\KeywordTok{write.csv}\NormalTok{(x, }\DataTypeTok{file =} \StringTok{"foo.csv"}\NormalTok{)}
\end{Highlighting}
\end{Shaded}

\newpage
\singlespacing 
\end{document}