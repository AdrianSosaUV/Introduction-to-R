\documentclass[11pt,]{article}
\usepackage[left=1in,top=1in,right=1in,bottom=1in]{geometry}
\newcommand*{\authorfont}{\fontfamily{phv}\selectfont}
\usepackage[]{mathpazo}


  \usepackage[T1]{fontenc}
  \usepackage[utf8]{inputenc}




\usepackage{abstract}
\renewcommand{\abstractname}{}    % clear the title
\renewcommand{\absnamepos}{empty} % originally center

\renewenvironment{abstract}
 {{%
    \setlength{\leftmargin}{0mm}
    \setlength{\rightmargin}{\leftmargin}%
  }%
  \relax}
 {\endlist}

\makeatletter
\def\@maketitle{%
  \newpage
%  \null
%  \vskip 2em%
%  \begin{center}%
  \let \footnote \thanks
    {\fontsize{18}{20}\selectfont\raggedright  \setlength{\parindent}{0pt} \@title \par}%
}
%\fi
\makeatother




\setcounter{secnumdepth}{0}

\usepackage{color}
\usepackage{fancyvrb}
\newcommand{\VerbBar}{|}
\newcommand{\VERB}{\Verb[commandchars=\\\{\}]}
\DefineVerbatimEnvironment{Highlighting}{Verbatim}{commandchars=\\\{\}}
% Add ',fontsize=\small' for more characters per line
\usepackage{framed}
\definecolor{shadecolor}{RGB}{248,248,248}
\newenvironment{Shaded}{\begin{snugshade}}{\end{snugshade}}
\newcommand{\AlertTok}[1]{\textcolor[rgb]{0.94,0.16,0.16}{#1}}
\newcommand{\AnnotationTok}[1]{\textcolor[rgb]{0.56,0.35,0.01}{\textbf{\textit{#1}}}}
\newcommand{\AttributeTok}[1]{\textcolor[rgb]{0.77,0.63,0.00}{#1}}
\newcommand{\BaseNTok}[1]{\textcolor[rgb]{0.00,0.00,0.81}{#1}}
\newcommand{\BuiltInTok}[1]{#1}
\newcommand{\CharTok}[1]{\textcolor[rgb]{0.31,0.60,0.02}{#1}}
\newcommand{\CommentTok}[1]{\textcolor[rgb]{0.56,0.35,0.01}{\textit{#1}}}
\newcommand{\CommentVarTok}[1]{\textcolor[rgb]{0.56,0.35,0.01}{\textbf{\textit{#1}}}}
\newcommand{\ConstantTok}[1]{\textcolor[rgb]{0.00,0.00,0.00}{#1}}
\newcommand{\ControlFlowTok}[1]{\textcolor[rgb]{0.13,0.29,0.53}{\textbf{#1}}}
\newcommand{\DataTypeTok}[1]{\textcolor[rgb]{0.13,0.29,0.53}{#1}}
\newcommand{\DecValTok}[1]{\textcolor[rgb]{0.00,0.00,0.81}{#1}}
\newcommand{\DocumentationTok}[1]{\textcolor[rgb]{0.56,0.35,0.01}{\textbf{\textit{#1}}}}
\newcommand{\ErrorTok}[1]{\textcolor[rgb]{0.64,0.00,0.00}{\textbf{#1}}}
\newcommand{\ExtensionTok}[1]{#1}
\newcommand{\FloatTok}[1]{\textcolor[rgb]{0.00,0.00,0.81}{#1}}
\newcommand{\FunctionTok}[1]{\textcolor[rgb]{0.00,0.00,0.00}{#1}}
\newcommand{\ImportTok}[1]{#1}
\newcommand{\InformationTok}[1]{\textcolor[rgb]{0.56,0.35,0.01}{\textbf{\textit{#1}}}}
\newcommand{\KeywordTok}[1]{\textcolor[rgb]{0.13,0.29,0.53}{\textbf{#1}}}
\newcommand{\NormalTok}[1]{#1}
\newcommand{\OperatorTok}[1]{\textcolor[rgb]{0.81,0.36,0.00}{\textbf{#1}}}
\newcommand{\OtherTok}[1]{\textcolor[rgb]{0.56,0.35,0.01}{#1}}
\newcommand{\PreprocessorTok}[1]{\textcolor[rgb]{0.56,0.35,0.01}{\textit{#1}}}
\newcommand{\RegionMarkerTok}[1]{#1}
\newcommand{\SpecialCharTok}[1]{\textcolor[rgb]{0.00,0.00,0.00}{#1}}
\newcommand{\SpecialStringTok}[1]{\textcolor[rgb]{0.31,0.60,0.02}{#1}}
\newcommand{\StringTok}[1]{\textcolor[rgb]{0.31,0.60,0.02}{#1}}
\newcommand{\VariableTok}[1]{\textcolor[rgb]{0.00,0.00,0.00}{#1}}
\newcommand{\VerbatimStringTok}[1]{\textcolor[rgb]{0.31,0.60,0.02}{#1}}
\newcommand{\WarningTok}[1]{\textcolor[rgb]{0.56,0.35,0.01}{\textbf{\textit{#1}}}}



\title{Programación Estadística: Arreglos  }



\author{\Large Adrián Sosa\vspace{0.05in} \newline\normalsize\emph{}   \and \Large \vspace{0.05in} \newline\normalsize\emph{Universidad Veracruzana}  }


\date{}

\usepackage{titlesec}

\titleformat*{\section}{\normalsize\bfseries}
\titleformat*{\subsection}{\normalsize\itshape}
\titleformat*{\subsubsection}{\normalsize\itshape}
\titleformat*{\paragraph}{\normalsize\itshape}
\titleformat*{\subparagraph}{\normalsize\itshape}


\usepackage{natbib}
\bibliographystyle{plainnat}
\usepackage[strings]{underscore} % protect underscores in most circumstances



\newtheorem{hypothesis}{Hypothesis}
\usepackage{setspace}


% set default figure placement to htbp
\makeatletter
\def\fps@figure{htbp}
\makeatother

\usepackage{hyperref}

% move the hyperref stuff down here, after header-includes, to allow for - \usepackage{hyperref}

\makeatletter
\@ifpackageloaded{hyperref}{}{%
\ifxetex
  \PassOptionsToPackage{hyphens}{url}\usepackage[setpagesize=false, % page size defined by xetex
              unicode=false, % unicode breaks when used with xetex
              xetex]{hyperref}
\else
  \PassOptionsToPackage{hyphens}{url}\usepackage[draft,unicode=true]{hyperref}
\fi
}

\@ifpackageloaded{color}{
    \PassOptionsToPackage{usenames,dvipsnames}{color}
}{%
    \usepackage[usenames,dvipsnames]{color}
}
\makeatother
\hypersetup{breaklinks=true,
            bookmarks=true,
            pdfauthor={Adrián Sosa () and  (Universidad Veracruzana)},
             pdfkeywords = {},  
            pdftitle={Programación Estadística: Arreglos},
            colorlinks=true,
            citecolor=blue,
            urlcolor=blue,
            linkcolor=magenta,
            pdfborder={0 0 0}}
\urlstyle{same}  % don't use monospace font for urls

% Add an option for endnotes. -----


% add tightlist ----------
\providecommand{\tightlist}{%
\setlength{\itemsep}{0pt}\setlength{\parskip}{0pt}}

% add some other packages ----------

% \usepackage{multicol}
% This should regulate where figures float
% See: https://tex.stackexchange.com/questions/2275/keeping-tables-figures-close-to-where-they-are-mentioned
\usepackage[section]{placeins}


\begin{document}
	
% \pagenumbering{arabic}% resets `page` counter to 1 
%
% \maketitle

{% \usefont{T1}{pnc}{m}{n}
\setlength{\parindent}{0pt}
\thispagestyle{plain}
{\fontsize{18}{20}\selectfont\raggedright 
\maketitle  % title \par  

}

{
   \vskip 13.5pt\relax \normalsize\fontsize{11}{12} 
\textbf{\authorfont Adrián Sosa} \hskip 15pt \emph{\small }   \par \textbf{\authorfont } \hskip 15pt \emph{\small Universidad Veracruzana}   

}

}






\vskip -8.5pt


 % removetitleabstract

\noindent  

\hypertarget{arreglos}{%
\section{Arreglos}\label{arreglos}}

Existen diferentes tipos de Arreglos datos los cuales se mencionan a
continuación:

\begin{verbatim}
* Vector
* Matriz
* Data frame
* Lista
* Series de tiempo
* Expresiones
\end{verbatim}

\hypertarget{vector}{%
\subsection{Vector}\label{vector}}

Es un conjunto de datos ya sea númericos, lógicos o de caracter
dependiendo como sean especificados en el argumento \emph{mode}, conta
de dos parametros ``mode'' y ``length'', este ultimo define la longitud
del vector.

\begin{Shaded}
\begin{Highlighting}[]
\CommentTok{# Vector(mode="logical", length=0)}
\NormalTok{x <-}\StringTok{ }\KeywordTok{vector}\NormalTok{(}\DataTypeTok{mode=}\StringTok{"logical"}\NormalTok{, }\DataTypeTok{length=}\DecValTok{3}\NormalTok{)}
\KeywordTok{print}\NormalTok{(x)}
\end{Highlighting}
\end{Shaded}

\begin{verbatim}
## [1] FALSE FALSE FALSE
\end{verbatim}

El argumento \emph{mode} puede adquirir los siguientes valores:

\begin{verbatim}
* any
* list
* expression
* symbol
* pairlist
\end{verbatim}

También puede ser utilizado en operadores lógicos de la siguiente
manera:

\begin{Shaded}
\begin{Highlighting}[]
\KeywordTok{as.vector}\NormalTok{(x, }\DataTypeTok{mode =} \StringTok{"any"}\NormalTok{)}
\end{Highlighting}
\end{Shaded}

\begin{verbatim}
## [1] FALSE FALSE FALSE
\end{verbatim}

\begin{Shaded}
\begin{Highlighting}[]
\KeywordTok{is.vector}\NormalTok{(x, }\DataTypeTok{mode =} \StringTok{"any"}\NormalTok{)}
\end{Highlighting}
\end{Shaded}

\begin{verbatim}
## [1] TRUE
\end{verbatim}

\begin{center}\rule{0.5\linewidth}{0.5pt}\end{center}

\hypertarget{matriz}{%
\subsection{Matriz}\label{matriz}}

Una matriz es un vector con un atributo adicional (dim) el cual a su ves
es un vector númerico de logitud 2, que define el número de filas y
columnas de la matriz, el monado para quear este tipo de datos es
\emph{matrix}:

\begin{Shaded}
\begin{Highlighting}[]
\CommentTok{# matrix(data = NA, nrow = 1, ncol = 1, byrow = FALSE, dimnames = NULL)}
\NormalTok{x <-}\StringTok{ }\KeywordTok{matrix}\NormalTok{(}\DataTypeTok{data =} \OtherTok{NA}\NormalTok{, }\DataTypeTok{nrow =} \DecValTok{2}\NormalTok{, }\DataTypeTok{ncol =} \DecValTok{2}\NormalTok{, }\DataTypeTok{byrow =} \OtherTok{FALSE}\NormalTok{, }\DataTypeTok{dimnames =} \OtherTok{NULL}\NormalTok{)}
\KeywordTok{print}\NormalTok{(x)}
\end{Highlighting}
\end{Shaded}

\begin{verbatim}
##      [,1] [,2]
## [1,]   NA   NA
## [2,]   NA   NA
\end{verbatim}

Los argumentos operan de la siguiente manera:

\begin{verbatim}
* data - Recibe la información que formara parte de la matriz.
* nrow - Número de filas. 
* ncol - número de columnas.
* byrow - indica si los valores en data deben llenar las columnas sucesivamente(FALSE) o las filas(TRUE)
* dimnames - permite asignar nombres a las filas y columnas.
\end{verbatim}

\begin{Shaded}
\begin{Highlighting}[]
\NormalTok{x <-}\StringTok{ }\KeywordTok{matrix}\NormalTok{(}\DataTypeTok{data =} \DecValTok{1}\OperatorTok{:}\DecValTok{15}\NormalTok{, }\DataTypeTok{nrow =} \DecValTok{5}\NormalTok{, }\DataTypeTok{ncol =} \DecValTok{3}\NormalTok{, }\DataTypeTok{byrow =} \OtherTok{FALSE}\NormalTok{, }\DataTypeTok{dimnames=} \KeywordTok{list}\NormalTok{(}\KeywordTok{c}\NormalTok{(}\StringTok{"row1"}\NormalTok{,}\StringTok{"row2"}\NormalTok{,}\StringTok{"row3"}\NormalTok{,}\StringTok{"row4"}\NormalTok{,}\StringTok{"row5"}\NormalTok{),}\KeywordTok{c}\NormalTok{(}\StringTok{"C1"}\NormalTok{,}\StringTok{"C2"}\NormalTok{,}\StringTok{"C3"}\NormalTok{)))}
\KeywordTok{print}\NormalTok{(x)}
\end{Highlighting}
\end{Shaded}

\begin{verbatim}
##      C1 C2 C3
## row1  1  6 11
## row2  2  7 12
## row3  3  8 13
## row4  4  9 14
## row5  5 10 15
\end{verbatim}

\begin{Shaded}
\begin{Highlighting}[]
\NormalTok{x <-}\StringTok{ }\KeywordTok{matrix}\NormalTok{(}\DataTypeTok{data =} \DecValTok{1}\OperatorTok{:}\DecValTok{15}\NormalTok{, }\DataTypeTok{nrow =} \DecValTok{5}\NormalTok{, }\DataTypeTok{ncol =} \DecValTok{3}\NormalTok{, }\DataTypeTok{byrow =} \OtherTok{TRUE}\NormalTok{, }\DataTypeTok{dimnames=} \KeywordTok{list}\NormalTok{(}\KeywordTok{c}\NormalTok{(}\StringTok{"row1"}\NormalTok{,}\StringTok{"row2"}\NormalTok{,}\StringTok{"row3"}\NormalTok{,}\StringTok{"row4"}\NormalTok{,}\StringTok{"row5"}\NormalTok{),}\KeywordTok{c}\NormalTok{(}\StringTok{"C1"}\NormalTok{,}\StringTok{"C2"}\NormalTok{,}\StringTok{"C3"}\NormalTok{)))}
\KeywordTok{print}\NormalTok{(x)}
\end{Highlighting}
\end{Shaded}

\begin{verbatim}
##      C1 C2 C3
## row1  1  2  3
## row2  4  5  6
## row3  7  8  9
## row4 10 11 12
## row5 13 14 15
\end{verbatim}

\hypertarget{operaciones-con-matrices}{%
\subsubsection{Operaciones con
matrices}\label{operaciones-con-matrices}}

\begin{Shaded}
\begin{Highlighting}[]
\NormalTok{x <-}\StringTok{ }\KeywordTok{matrix}\NormalTok{(}\DecValTok{1}\OperatorTok{:}\DecValTok{25}\NormalTok{, }\DecValTok{5}\NormalTok{, }\DecValTok{5}\NormalTok{, }\OtherTok{FALSE}\NormalTok{)}
\NormalTok{y <-}\StringTok{ }\KeywordTok{matrix}\NormalTok{(}\DecValTok{1}\OperatorTok{:}\DecValTok{25}\NormalTok{, }\DecValTok{5}\NormalTok{, }\DecValTok{5}\NormalTok{, }\OtherTok{TRUE}\NormalTok{)}
\NormalTok{x}
\end{Highlighting}
\end{Shaded}

\begin{verbatim}
##      [,1] [,2] [,3] [,4] [,5]
## [1,]    1    6   11   16   21
## [2,]    2    7   12   17   22
## [3,]    3    8   13   18   23
## [4,]    4    9   14   19   24
## [5,]    5   10   15   20   25
\end{verbatim}

\begin{Shaded}
\begin{Highlighting}[]
\NormalTok{y}
\end{Highlighting}
\end{Shaded}

\begin{verbatim}
##      [,1] [,2] [,3] [,4] [,5]
## [1,]    1    2    3    4    5
## [2,]    6    7    8    9   10
## [3,]   11   12   13   14   15
## [4,]   16   17   18   19   20
## [5,]   21   22   23   24   25
\end{verbatim}

\hypertarget{suma}{%
\paragraph{Suma}\label{suma}}

\begin{Shaded}
\begin{Highlighting}[]
\NormalTok{x }\OperatorTok{+}\StringTok{ }\NormalTok{y}
\end{Highlighting}
\end{Shaded}

\begin{verbatim}
##      [,1] [,2] [,3] [,4] [,5]
## [1,]    2    8   14   20   26
## [2,]    8   14   20   26   32
## [3,]   14   20   26   32   38
## [4,]   20   26   32   38   44
## [5,]   26   32   38   44   50
\end{verbatim}

\hypertarget{resta}{%
\paragraph{Resta}\label{resta}}

\begin{Shaded}
\begin{Highlighting}[]
\NormalTok{x }\OperatorTok{-}\StringTok{ }\NormalTok{y}
\end{Highlighting}
\end{Shaded}

\begin{verbatim}
##      [,1] [,2] [,3] [,4] [,5]
## [1,]    0    4    8   12   16
## [2,]   -4    0    4    8   12
## [3,]   -8   -4    0    4    8
## [4,]  -12   -8   -4    0    4
## [5,]  -16  -12   -8   -4    0
\end{verbatim}

\hypertarget{multiplicaciuxf3n}{%
\paragraph{Multiplicación}\label{multiplicaciuxf3n}}

\begin{Shaded}
\begin{Highlighting}[]
\NormalTok{x }\OperatorTok{*}\StringTok{ }\NormalTok{y}
\end{Highlighting}
\end{Shaded}

\begin{verbatim}
##      [,1] [,2] [,3] [,4] [,5]
## [1,]    1   12   33   64  105
## [2,]   12   49   96  153  220
## [3,]   33   96  169  252  345
## [4,]   64  153  252  361  480
## [5,]  105  220  345  480  625
\end{verbatim}

\hypertarget{diviciuxf3n}{%
\paragraph{Divición}\label{diviciuxf3n}}

\begin{Shaded}
\begin{Highlighting}[]
\NormalTok{x }\OperatorTok{/}\StringTok{ }\NormalTok{y}
\end{Highlighting}
\end{Shaded}

\begin{verbatim}
##           [,1]      [,2]      [,3]      [,4]     [,5]
## [1,] 1.0000000 3.0000000 3.6666667 4.0000000 4.200000
## [2,] 0.3333333 1.0000000 1.5000000 1.8888889 2.200000
## [3,] 0.2727273 0.6666667 1.0000000 1.2857143 1.533333
## [4,] 0.2500000 0.5294118 0.7777778 1.0000000 1.200000
## [5,] 0.2380952 0.4545455 0.6521739 0.8333333 1.000000
\end{verbatim}

\begin{center}\rule{0.5\linewidth}{0.5pt}\end{center}

\hypertarget{marco-de-datos}{%
\subsection{Marco de datos}\label{marco-de-datos}}

Un marco de datos o ``Data.frame'' se crea de manera implítica con la
función \emph{read.table}, de igual manera es posible hacerlo con la
función \emph{data.frame}.

\begin{Shaded}
\begin{Highlighting}[]
\NormalTok{x <-}\StringTok{ }\DecValTok{1}\OperatorTok{:}\DecValTok{4}
\NormalTok{n <-}\StringTok{ }\DecValTok{10}
\KeywordTok{data.frame}\NormalTok{(x,n)}
\end{Highlighting}
\end{Shaded}

\begin{verbatim}
##   x  n
## 1 1 10
## 2 2 10
## 3 3 10
## 4 4 10
\end{verbatim}

\begin{center}\rule{0.5\linewidth}{0.5pt}\end{center}

\hypertarget{lista}{%
\subsection{Lista}\label{lista}}

Una lista se crea de manera similar a un marco de datos por medio d ela
función \emph{list}, puede incluir cualquier tipo de objetos, a
diferencia del \emph{data.frame} los nombres de los objetos no se toman
por defecto.

\begin{Shaded}
\begin{Highlighting}[]
\NormalTok{L <-}\StringTok{ }\KeywordTok{list}\NormalTok{(x,n)}
\NormalTok{L}
\end{Highlighting}
\end{Shaded}

\begin{verbatim}
## [[1]]
## [1] 1 2 3 4
## 
## [[2]]
## [1] 10
\end{verbatim}

\begin{center}\rule{0.5\linewidth}{0.5pt}\end{center}

\hypertarget{series-de-tiempo}{%
\subsection{Series de tiempo}\label{series-de-tiempo}}

La función \emph{ts} crea un objeto de clase ``ts'' (serie de tiempo) a
partir de un vector( serie de tiempo única) o una matriz(serie
multivariada). los argumentos que recibe son:

\begin{verbatim}
ts(data= NA, start=1, end= numeric(0), frecuency=1, detat= 1, ts.sps= 
getOption("ts.eps"), class, names)

    * data      -   Un Vector o matriz
    * start     -   El tiempo en la primera observación ya sea un número o un vector
                    con dos enteros
    * end       -   El tiempo de la última observación especificado de la misma 
                    manera que *start*
    * frequency -   El número de observaciones por unidad de tiempo
    * deltat    -   Fracción del periodo de muestreo entre observaciones sucesivas 
                    **solo especificar "frequency" o "deltat" **   
    * ts.eps    -   Tolerancia para la comparación de series
    * class     -   Clase asignada el objeto
    * names     -   Para una serie multivariada, un vector de tipo caracter con los
                    nombres de las series individuales
\end{verbatim}

\begin{Shaded}
\begin{Highlighting}[]
\KeywordTok{ts}\NormalTok{(}\DecValTok{1}\OperatorTok{:}\DecValTok{10}\NormalTok{, }\DataTypeTok{start=}\DecValTok{1959}\NormalTok{)}
\end{Highlighting}
\end{Shaded}

\begin{verbatim}
## Time Series:
## Start = 1959 
## End = 1968 
## Frequency = 1 
##  [1]  1  2  3  4  5  6  7  8  9 10
\end{verbatim}

\begin{Shaded}
\begin{Highlighting}[]
\KeywordTok{ts}\NormalTok{(}\DecValTok{1}\OperatorTok{:}\DecValTok{50}\NormalTok{, }\DataTypeTok{frequency=} \DecValTok{12}\NormalTok{, }\DataTypeTok{start=}\KeywordTok{c}\NormalTok{(}\DecValTok{1959}\NormalTok{,}\DecValTok{2}\NormalTok{))}
\end{Highlighting}
\end{Shaded}

\begin{verbatim}
##      Jan Feb Mar Apr May Jun Jul Aug Sep Oct Nov Dec
## 1959       1   2   3   4   5   6   7   8   9  10  11
## 1960  12  13  14  15  16  17  18  19  20  21  22  23
## 1961  24  25  26  27  28  29  30  31  32  33  34  35
## 1962  36  37  38  39  40  41  42  43  44  45  46  47
## 1963  48  49  50
\end{verbatim}

\begin{Shaded}
\begin{Highlighting}[]
\KeywordTok{ts}\NormalTok{(}\DecValTok{1}\OperatorTok{:}\DecValTok{10}\NormalTok{, }\DataTypeTok{frequency=} \DecValTok{4}\NormalTok{, }\DataTypeTok{start=}\KeywordTok{c}\NormalTok{(}\DecValTok{1959}\NormalTok{,}\DecValTok{2}\NormalTok{))}
\end{Highlighting}
\end{Shaded}

\begin{verbatim}
##      Qtr1 Qtr2 Qtr3 Qtr4
## 1959         1    2    3
## 1960    4    5    6    7
## 1961    8    9   10
\end{verbatim}

\begin{center}\rule{0.5\linewidth}{0.5pt}\end{center}

\hypertarget{expresiones}{%
\subsection{Expresiones}\label{expresiones}}

Los objetos de la clase \emph{expresión} son de gran importancia en R.
Euna \emph{expresión} es una serie de caraacteres que hacen sentido para
R, los comandos de R son puramente expresiónes, cuando se escribe un
comando este es evaluado por R y ejecutado si resulta válido.

\begin{Shaded}
\begin{Highlighting}[]
\NormalTok{x <-}\StringTok{ }\DecValTok{3}\NormalTok{; y <-}\StringTok{ }\FloatTok{2.5}\NormalTok{; z <-}\StringTok{ }\DecValTok{1}
\NormalTok{exp1 <-}\StringTok{ }\KeywordTok{expression}\NormalTok{(x}\OperatorTok{/}\NormalTok{(y}\OperatorTok{+}\StringTok{ }\KeywordTok{exp}\NormalTok{(z)))}
\NormalTok{exp1}
\end{Highlighting}
\end{Shaded}

\begin{verbatim}
## expression(x/(y + exp(z)))
\end{verbatim}

\begin{Shaded}
\begin{Highlighting}[]
\KeywordTok{eval}\NormalTok{(exp1)}
\end{Highlighting}
\end{Shaded}

\begin{verbatim}
## [1] 0.5749019
\end{verbatim}

Las expresiones se pueden usar, entre otras cosas, para incluir
ecuaciones en graficos o como argumentos en ciertas funciones, por
ejemplo \emph{D()} que calcula derivadas aprciales:

\begin{Shaded}
\begin{Highlighting}[]
\KeywordTok{D}\NormalTok{(exp1,}\StringTok{"x"}\NormalTok{)}
\end{Highlighting}
\end{Shaded}

\begin{verbatim}
## 1/(y + exp(z))
\end{verbatim}

\begin{Shaded}
\begin{Highlighting}[]
\KeywordTok{D}\NormalTok{(exp1,}\StringTok{"y"}\NormalTok{)}
\end{Highlighting}
\end{Shaded}

\begin{verbatim}
## -(x/(y + exp(z))^2)
\end{verbatim}

\begin{Shaded}
\begin{Highlighting}[]
\KeywordTok{D}\NormalTok{(exp1,}\StringTok{"z"}\NormalTok{)}
\end{Highlighting}
\end{Shaded}

\begin{verbatim}
## -(x * exp(z)/(y + exp(z))^2)
\end{verbatim}

\begin{center}\rule{0.5\linewidth}{0.5pt}\end{center}

\newpage
\singlespacing 
\end{document}
